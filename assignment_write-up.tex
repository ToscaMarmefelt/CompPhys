%%% Preamble
\documentclass[paper=a4, fontsize=11pt]{scrartcl}
\usepackage[T1]{fontenc}
\usepackage{fourier}

\usepackage[english]{babel}															% English language/hyphenation
\usepackage[protrusion=true,expansion=true]{microtype}	
\usepackage{amsmath,amsfonts,amsthm} % Math packages
\usepackage[pdftex]{graphicx}	
\usepackage{url}


%%% Custom sectioning
\usepackage{sectsty}
\allsectionsfont{\centering \normalfont\scshape}


%%% Custom headers/footers (fancyhdr package)
\usepackage{fancyhdr}
\pagestyle{fancyplain}
\fancyhead{}											% No page header
\fancyfoot[L]{}											% Empty 
\fancyfoot[C]{}											% Empty
\fancyfoot[R]{\thepage}									% Pagenumbering
\renewcommand{\headrulewidth}{0pt}			% Remove header underlines
\renewcommand{\footrulewidth}{0pt}				% Remove footer underlines
\setlength{\headheight}{13.6pt}

\usepackage{bbold}


%%% Equation and float numbering
%\numberwithin{equation}{section}		% Equationnumbering: section.eq#
%\numberwithin{figure}{section}			% Figurenumbering: section.fig#
%\numberwithin{table}{section}				% Tablenumbering: section.tab#


%%% Maketitle metadata
\newcommand{\horrule}[1]{\rule{\linewidth}{#1}} 	% Horizontal rule

\title{
		%\vspace{-1in} 	
		\usefont{OT1}{bch}{b}{n}
		\normalfont \normalsize \textsc{Department of Physics $|$ Imperial College London} \\ [25pt]
		\horrule{0.5pt} \\[0.4cm]
		\huge Computational Physics: Assignment \\
		\horrule{2pt} \\[0.5cm]
}
\author{
		\normalfont 								\normalsize
        Tosca Cederberg Marmefelt\\[-3pt]		\normalsize
        CID: $01204044$\\[-3pt]		\normalsize
        \today
}
\date{}


%%% Begin document
\begin{document}
\maketitle
\section{Floating point variables}

\begin{center}
	\label{fact2}
	\begin{tabular}{l || c | c | c | c | c }
		\textbf{Precision} & \textbf{Sign} & \textbf{Mantissa} & \textbf{Exponent} & \textbf{Total} & \textbf{Numpy Data Type} \\ \hline \hline
		Half & $1$ bit & $10$ bits & $5$ bits & $16$ bits & \textit{float16} \\ \hline 
		Single & $1$ bit & $23$ bits & $8$ bits & $32$ bits & \textit{float32} \\ \hline 
		Double & $1$ bit & $52$ bits & $11$ bits & $64$ bits & \textit{float64} \\ \hline 
		Extended & $1$ bit & $112$ bits & $15$ bits & $128$ bits & Not available in Python
	\end{tabular}
\end{center}

\begin{center}
	\label{result2}
	\begin{tabular}{l || c | c}
	\textbf{Precision} & \textbf{Machine Accuracy}  & \textbf{Smallest Value Meaningfully Subtracted from $1$}\\ \hline \hline
	Half & $0.0009765625$ & $0.00048828125$ \\ \hline
	Single & $1.1920928955078125 \times 10^{-7}$ & $ 5.960464477539063 \times 10^{-8}$\\ \hline
	Dounble & $2.220446049250313 \times 10^{-16}$ & $1.1102230246251565 \times 10^{-16}$
	\end{tabular}
\end{center}
From Table \ref{result2}, it is clear that
%\begin{align} 
%	\begin{split}
%	(x+y)^3 	&= (x+y)^2(x+y)\\
%					&=(x^2+2xy+y^2)(x+y)\\
%					&=(x^3+2x^2y+xy^2) + (x^2y+2xy^2+y^3)\\
%					&=x^3+3x^2y+3xy^2+y^3
%	\end{split}					
%\end{align}


\section{Matrix methods}

\begin{align}
\label{A define}
	A = 
	\begin{bmatrix}
	3 & 1 & 0 & 0 & 0 \\
  	3 & 9 & 4 & 0 & 0 \\
  	0 & 9 & 20 & 10 & 0 \\
  	0 & 0 & -22 & 31 & -25 \\
  	0 & 0 & 0 & -55 & 60
	\end{bmatrix}
\end{align}

\begin{align}	
	A \equiv LU =
		\begin{bmatrix}
	3 & 1 & 0 & 0 & 0 \\
	1 & 8 & 4 & 0 & 0 \\
	0 & 1.13 & 15.5 & 10 & 0 \\
	0 & 0 & -1.42 & 45.2 & -25 \\
	0 & 0 & 0 & -1.22 & 29.6
	\end{bmatrix}
\end{align}

\begin{equation}
\text{det}(A) = answer
\end{equation}

\begin{align}
\label{matrix equation}
A \textbf{x} \equiv LU \textbf{x} = 
	\begin{bmatrix}
	2 \\
	5 \\
	-4 \\
	8 \\
	9
	\end{bmatrix}
\end{align}

In order to calculate the inverse of our matrix $A$, we recall the fact that
\begin{equation}
A^{-1} A = \mathbb{1} \text{ ,}
\end{equation}
where $\mathbb{1}$ is the $N \times N$ identity matrix. For the matrix defined in \ref{A define}, this allows us to set up $N$ equations of the type
\begin{equation}
A^{-1}\text{[\textit{n}] } LU = \mathbb{1}\text{[\textit{n}]} \text{ ,}
\end{equation}
for each of the $n = 0, 1, ... , (N-1)$ columns in $A^{-1}$. Recognising that this is the same type of equation as that set up in Equation \ref{matrix equation}, we can use the same routine to loop through each column in $A^{-1}$ to find 

\begin{align}
\label{A define}
A^{-1} = 
\begin{bmatrix}
3 & 1 & 0 & 0 & 0 \\
3 & 9 & 4 & 0 & 0 \\
0 & 9 & 20 & 10 & 0 \\
0 & 0 & -22 & 31 & -25 \\
0 & 0 & 0 & -55 & 60
\end{bmatrix} \text{ . enter correct matrix here}
\end{align}

\section{Interpolation}


\section{Fourier transforms}


\section{Random numbers}
%%% End document
\end{document}